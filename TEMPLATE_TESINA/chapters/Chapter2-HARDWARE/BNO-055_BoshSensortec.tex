%%%%%%%%%%%%%%%%%%%%%%%%%%%%%%%%%%%%%%%%%%%%%%%%%%%%%%%%%%%%%%%%%%%
% TO DO LIST SPIEGAZIONE DEL BOSH SENSORTEC 
% Presentazione del sensore nell'ambito del progetto 
% Presentazione del sensore generale  [OK]
% Principio di misura del sensore: accelerometro, giroscopio e magnetometro [OK]
% Dati tecnici: Dimensioni, peso, alimentaizione, collegamenti, comunicazione 
% Modalità operative del sensore spingendo sulla sensor fusion 

\section{BNO-055 Bosh Sensortec. Unità di misura inerziale e magnetometro}
\textcolor{red}{a cosa serve l'imu in questo progetto?}\\
Il \textbf{BNO-055} è un sensore di orientamento assoluto a 9 assi sviluppato da \textbf{Bosh Sensortec}.
Il dispositivo è un \textit{System in Package}\textcolor{blue}{[A]} che integra un accelerometro triassiale a 14 bit, un giroscopio triassiale a 16 bit, un sensore geomagnetico triassiale e un
microcontrollore Cortex $M0^+$ a 32 bit incaricato nell'eseguire il \textit{software di sensor fusion} integrato.\\
Il \textit{software}, poc'anzi citato, combina i dati provenienti da accelerometro, giroscopio e magnetometro, fornendo: \textbf{Quaternioni, Angoli di Eulero come pitch, roll e yaw e Vettori di orientamento lineari e gravitazionali}.
Nel presente progetto, il \textit{chip} BNO-055 è stato acquistato con montaggio su \textit{scheda di breakout} incluso, dunque nel paragrafo sottostante sono stae aggiunte sia le dimensioni/peso del modulo che le dimensioni/peso del chip.

\subsection{BNO-055 BoshSensortec. Catatteristiche e archiettura}
A seguire una rappresentazione schematica del \textit{System in package BNO-055 Bosh Sensortec}
\begin{figure}[htbp]
    \centering
    \includegraphics[width=0.5\textwidth]{chapters/Chapter2-HARDWARE/Figures/IMG3-Architettura BNO-055.png}
    \caption{\textcolor{black}{Architettura di sistema}}
    \label{fig:etichetta}
\end{figure}



\paragraph{\small{Parametri fisici}}
\begin{itemize}
    \item dimensioni del modulo: $20\times24\times2$ mm.
    \item dimensioni del chip: $3.8\times5.2\times1.1$ mm.
    \item massa del modulo: 3 g.
    \item massa del chip: $\approx 150 mg$.
\end{itemize}

\paragraph{\small{Alimentazione}}\mbox{}\\
Il sensore ha due distinti ingressi di alimentazione:
\begin{itemize}
    \item $V_{DD}$ è il principale \textit{pin} di alimentazione per i sensori interni.
    \item $V_{DDIO}$ è l'ingresso distinto di alimentazione del \textit{$\mu C$} e delle interfacce digitali.
\end{itemize}
Al fine di ottenere una corretta alimentazione per il \textit{System in package} è necessario che l'ingresso $V_{DD}$ sia sotto tensione consigliata prima di alimentare anche l'ingresso $V_{DDIO}$.\\
Il BNO-055 supporta tre differenti modalità di alimentazione: \textit{Normal mode}, \textit{Low power mode} and \textit{Suspend mode}. Per lo sviluppo è stata scelta la modalità \textit{Normal mode}.\\
\small{\textit{Per approdondire le altre modalità, consultare il datasheet: \textcolor{red}{[Riferimento]}}}

\paragraph{\small{Protocolli di comunicazione disponibili}}
\begin{itemize}
    \item $I^2C$.
    \item UART.
    \item SPI.
\end{itemize}
%Anche se lo scrivi sopra nella parte (a cosa serve l'imu a questo progetto rammenta)
\textbf{\textit{\small{Si rammenta che nel presente progetto è stato utilizzato il protoccolo di comunicazione $I^2C$}}}

\subsection{Principio di misura dell'accelerometro, giroscopio e magnetometro}
\paragraph{Accelerometro}\mbox{}\\
L'accelerometro integrato nel BNO-055 Bosh Sensortec è di tipo capacitivo M.E.M.S, \textit{Micro-Eletro-Mechanical System}. Il dispositivo consente la misura dell'accelerazione lungo i tre assi cartesiani.\\
Il principio di funzionamento si basa sulla rilevazione delle variazioni di capacità tra microstrutture mobili e fisse realizzate su un substrato di silicio.\\
L'elemento sensibile di ciascun asse è costituito da una massa sospesa, \textit{proof mass}, vincolata da microtravi elastiche ad una cornice ancorata al substrato. In condizioni di quiete, la massa è in equilibrio e la capacità tra le piastre interdigitate\textcolor{blue}{[A]} rimangono simmetriche.\\
Quando il dispositivo è sottoposto ad un'accelerazione lungo uno degli assi sensibili, la massa inerziale si sposta in direzione opposta a quella dell'accelerazione, generando una variazione differenziale delle capacità tra le piastre.\\
La variazione di capacità viene rilevata da un circuito integrato, che la converte in un segnale elettrico, proporzionale alla velocità applicata.\\
Il segnale analogico prodotto, viene successivamente digitalizzato da un convertirore analogico digitale integrato nel chip.

\paragraph{Giroscopio}\mbox{}\\
Il giroscopio integrato nel dispositivo fa parte della categoria M.E.M.S di tipo vibrante, \textit{vibrating structure gyroscope}, e consente la misura della velocità angolare lungo i tre assi cartesiani.\\
Il principio di funzionamento si basa sull'effetto Coriolis, che si manifesta quando una massa in moto oscillatorio subisce una rotazione rispetto ad un sistema di riferimento inerziale.
All'interno del sensore, ciascun asse dispone di una o più masse vibranti, le quali vengono mantenute in oscillazione a frequenza costante, mediante un circuito di attuazione elettrostatica. Quando il dispositivo ruota attorno ad uno degli assi
la massa subisce una forza di Coriolis data da:
\begin{equation}
    \vec{F_c}=2m(\vec{v}\times \vec{\omega }) 
\end{equation}
dove $m$ è la massa oscillante, $\vec{v}$ è la velocità della massa nella sua traiettoria vibrante, $\vec{\omega}$ è la velocità angolare del corpo.\\
Questa forza induce uno spostamento trasversale rispetto alla direzione di vibrazione, che viene rilevato attraverso variazioni di capacità tra elettrodi fissi e mobili, similmente a quanto avviene nell'accelerometro.\\
Tali variazioni, proporzionali alla velocità angolare, vengono convertite in un segnale elettrico mediante un circuito di lettura differenziale e successivamente digitalizzate tramite un convertitore analogico digitale integrato.

\paragraph{Magnetometro}\mbox{}\\
Il sensore integra inoltre un \textbf{magnetometro ad effetto Hall}, sfruttante il fenomeno fisico della tensione di Hall per misurare l'orientamento del dispositivo rispetto al campo magnetico terrestre.\\
Nella sua forma più semplice, un elemento di Hall è una sottilissima lamina di materiale conduttore o semiconduttore attraversata da una corrente continua controllata. Se su questa lamina agisce un campo magnetico con componente perpendicolare alla direzione della corrente,
i portatori di carica vengono deviati lateralmente della \textbf{forza di Lorentz}:
\begin{equation}
    F = q(\vec{v}\times\vec{B})
\end{equation}
con $\vec{v}$ vettore velocità della carica e $\vec{B}$ vettore campo magnetico. Questa deviazione procude un accumulo di carica ai bordi opposti della lamina, e, in regime stazionario, un campo elettrico trasversale 
che equilibria la forza magnetica. Il risultato è una differenza di potenziale trasversale, la \textbf{tensione di Hall}: $V_H$ che risulta proporzionale alla componente di campo magnetico normale alla superficie del 
sensore e alla corrente che lo attraversa. Per un elmento \textit{Hall} omogeneo abbiamo:
\begin{equation}
    V_H = \frac{I\vec{B_{\perp}}}{nqs}
\end{equation}
dove $\vec{B_{\perp}}$ è la componente del campo perpendicolare al piano della lamina, $n$ la densità di portatori, $q$ il valore di carica elementare e $s$ lo spessore del \textit{film} attivo.
Nel magnetometro ogni elemento di Hall è disposto in modo da avere la propria normale allineata con uno degli assi cartesiani del sistema di riferimento del sensore. Quando il campo magnetico terrestre
attraversa il dispositivo, ciascun elemento rileva la proiezione del vettore $\vec{B}$ lungo il proprio asse, trasformandola in una tensione di Hall proporzionale.\\
Dopo un'opportuna amplificazione della tensione di Hall percepita e una conversione analogico-digitale si ottiene un tripletto di valori numerici che rappresentano le tre componenti cartesiane del campo magnetico locale.\\
Il vettore tridimensionale ottenuto viene ricostruito e poi confrontato con il valore atteso del campo geomagnetico.\\
Attraverso il \textit{sensor fusion}, il magnetometro fornisce quindi il riferimento assoluto per l'\textit{azimut}, vale a dire, la direzione del Nord magnetico rispetto al sistema di riferimento solidale del sensore.

\subsection{Acceletometro, giroscopio e magnetometro}
L'accelerometro, il giroscopio e il magnetometro sono tutti prodotti da Bosh Sensortec.

\paragraph{\small{Accelerometro}}
\paragraph{\small{Giroscopio}}
\paragraph{\small{Magnetometro}}

\paragraph{Modalità operative}\mbox{}\\
Il dispositivo di misura fornisce una grande varietà di segnali di \textit{output}, che possono essere scelti selezziando l'appropiata modalità operativa.\\
Le modalità operative vengono classificate in base all'attivazione o meno del \textit{software di sensor fusion}, nello specifico distinguiamo tra le \textit{Non-Fusion modes} e le \textit{Fusion modes}.

\paragraph{Non-Fusion Modes}
\begin{itemize}
    \item \textbf{ACCONLY}: il dispositivo fornisce solamente dati grezzi di provenienti dall'accelerometro.
\end{itemize}
\textcolor{red}{CONTINUARE DA QUI}


