\section{Turnigy AereoDrive SK3-3536 1400KV}
Il Turnigy AereoDrive SK3 3536 1400kv è un motore \textit{Brushless} alimentato a corrente continua e pilotato a trifase con correnti alternate.\\
Il rotore contiene magneti permanenti disposti con polarità alternata lungo la sua circonferenza. Lo statore invece, ospita tre avvolgimenti elettrici (fasi) disposti generalmente a 120 gradi elettrici l'uno rispetto l'altro, connessi ai tre terminali d'uscita che si identificano come A,B e C.\\
Il rotore, essendo formato da magneti permanenti, non necessita di alimentazione: è lo statore a generare il campo magnetico rotante che induce il moto.\\
Il cambio di polarità degli avvolgimenti deve essere gestito da un sistema elettronico esterno, \textit{l'Elettronic Speed Controller (ESC)}, che nella successiva sezione verrà approfondito.\\
I tre conduttori che fuoriescono dal motore rappresentano i terminali delle tre fasi dello statore. Ciascun filo è collegato a uno degli avvolgimenti, e nessuno di essi costituisce un riferimento comune.
Per far ruotare il motore è necessario che tali fasi vengano alimentate secondo una precisa sequenza temporale, generando un campo magnetico rotante. 
Se si dovesse scambiare due dei tre cavi, l'ordine di commutazione si invertirebbe con conseguente variazione del senso di rotazione.
\subsection{Turnigy AereoDrive SK3-3536 1400 KV.\@ Caratteristiche tecniche:}
\begin{itemize}
    \item Tensione operativa: $[11.1\div16.8]$ [V].
    \item Valore KV: $1400\,\,\,(\frac{RPM}{V})$.
    \item Potenza massima: $590$ [W].
    \item Corrente massima: $40$ [A].
    \item Resistenza interna: $[0.021\div0-025]$ [Ohm].
    \item Corrente a vuoto: $0.021$ [A].
    \item Diametro dell'albero motore: $5$ [mm].
    \item Peso: $111$ [g].
    \item Spaziatura fori di montaggio: $25\times25$ [mm].
    \item Connettori: \textit{bullet} da $3.5$ [mm].
    \item Numero dei poli: $12$.
\end{itemize}
Il valore "KV" è comune nel contesto del modellismo ed indica il numero di giri al minuto (RPM) che il motore compie per ogni \textit{volt} applicato a vuoto, in formula:
\begin{equation}
    RPM = KV\cdot V
\end{equation}
A tensioni di alimentazione maggiori rispetto a quelle specificate il motore rischia di assorbire troppa corrente e surriscaldarsi.\\
I giri al minuto massimi dipenderanno dalla tensione applicata e dalle dimensioni delle eliche.\\
Nel presente progetto, il motore in interesse deve essere alimentato a batteria, vale a dire in corrente continua, lasciando spazio alla necessità di un dispositivo che trasformi la tensione continua erogata dalle batterie in un segnale trifase a corrente alternata.
Il dispositivo in questione è il precedentemente citato \textit{Electronic Speed Controller}.

\begin{figure}[h]
    \centering
    \includegraphics[width=0.5\textwidth]{chapters/Chapter2-HARDWARE/Figures/TurnigyAereoDriveSK3-3536_1400KV.jpg}
    \caption{Turnigy Aereo Drive SK3 3536 1400 KV}
    \label{fig:Hardware drone}
\end{figure}

%\paragraph{Pilotaggio tramite ESC}\mbox{}\\
%Poiché la batteria fornisce corrente continua e il motore richiede correnti alternate sfasate è indispensabile l'uso di un ESC.\\
%L'ESC trasforma la tensione continua della batteria (14.8 [V] della batteira utilizzata nel progetto) in tensioni alterante sfasate di 120 gradi elettrici. Questo avviene tramite una rete di transistor di potenza (MOSFET) organizzata in un ponte trifase.\\
%L'ESC deve conoscere in ogni istante la posizione angolare del rotore per applicare alla fasi dello statore la sequenza corretta di correnti. L'informazione in questo tipo di applicazioni è spesso stimata indirettamente mediante la misura della forza controelettromotrice indotta dal rotore, evitando l'uso di sensori.\\
%L'ESC riceve un comando esterno (tipicamente un segnale PWM di tipo RC, con impulsi di durata compresa tra 1 [ms] e 2 [ms]). L'ESC agisce da \textit{blackbox} e proporzionalmente al PWM in ingresso modula velocità del motore.\\
%La coppia, e quindi la potenza erogata, dipendono dal duty cycle della modulazione della tensione fornita. 



