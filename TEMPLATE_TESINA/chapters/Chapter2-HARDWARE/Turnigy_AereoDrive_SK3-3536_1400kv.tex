\section{Turnigy AereoDrive SK3-3536 1400KV}
Il Turnigy AereoDrive SK3 3536 1400kv è un motore \textit{Brushless} alimentato a corrente continua e pilotato a trifase con correnti alternate.\\
Il rotore contiene magneti permanenti disposti con polarità alternata lungo la sua circonferenza. Lo statore invece, ospita tre avvolgimenti elettrici (fasi) disposti generalmente a 120 gradi elettrici l'uno rispetto l'altro, connessi ai tre terminali d'uscita che si identificano come A,B e C.\\
Il rotore, essendo formato da magneti permanenti, non necessita di alimentazione: è lo statore a generare il campo magnetico rotante che induce il moto.\\
Il cambio di polarità degli avvolgimenti deve essere gestito da un sistema elettronico esterno, \textit{l'Elettronic Speed Controller (ESC)}, che nella successiva sezione verrà approfondito.\\
I tre conduttori che fuoriescono dal motore rappresentano i terminali delle tre fasi dello statore. Ciascun filo è collegato a uno degli avvolgimenti, e nessuno di essi costituisce un riferimento comune.
Per far ruotare il motore è necessario che tali fasi vengano alimentate secondo una precisa sequenza temporale, generando un campo magnetico rotante. 
Se si dovesse scambiare due dei tre cavi, l'ordine di commutazione si invertirebbe con conseguente variazione del senso di rotazione.\\
Nel progetto sono stati impiegati due motori del tipo precedentemente citato, disposti in configurazione \textbf{controrotante}, con il fine di compensare le rotazioni indesiderate di \textbf{yaw}.
Tale scelta consente di bilanciare le coppie di reazione generate dalle eliche, riducendo il momento torcente sul velivolo. Inoltre, l'adozione di due motori si è resa necessaria in quanto un singolo
attuatore non sarebbe stato in grado di generare una spinta sufficiente a garantire la portanza richiesta dal sistema nelle condizioni operative previste.

\subsection{Turnigy AereoDrive SK3-3536 1400 KV.\@ Caratteristiche fisiche e tecniche:}
\begin{itemize}
    \item Tensione operativa: \textbf{[11.1$\div$16.8] [V]}.
    \item Valore KV: \textbf{1400 $(\frac{RPM}{V})$}.
    \item Potenza massima: \textbf{590 [W]}.
    \item Corrente massima: \textbf{40 [A]}.
    \item Resistenza interna: \textbf{[0.021$\div$0-025] [Ohm]}.
    \item Corrente a vuoto: \textbf{0.021 [A]}.
    \item Diametro dell'albero motore: \textbf{5 [mm]}.
    \item Peso: \textbf{111 [g]}.
    \item Spaziatura fori di montaggio: \textbf{25$\times$25 [mm]}.
    \item Connettori \textit{bullet} \textbf{da 3.5 [mm]}.
    \item Numero dei poli: \textbf{12}.
\end{itemize}

Il valore "KV" è comune nel contesto del modellismo ed indica il numero di giri al minuto (RPM) che il motore compie per ogni \textit{volt} applicato a vuoto, in formula:
\begin{equation}
    RPM = KV\cdot V
\end{equation}
A tensioni di alimentazione maggiori rispetto a quelle specificate il motore rischia di assorbire troppa corrente e surriscaldarsi.\\
I giri al minuto massimi dipenderanno dalla tensione applicata e dalle dimensioni delle eliche.\\
Nel presente progetto, il motore in interesse deve essere alimentato a batteria, vale a dire in corrente continua, lasciando spazio alla necessità di un dispositivo che trasformi la tensione continua erogata dalle batterie in un segnale trifase a corrente alternata.
Il dispositivo in questione è il precedentemente citato \textit{Electronic Speed Controller}.

\begin{figure}[h]
    \centering
    \includegraphics[width=0.5\textwidth]{chapters/Chapter2-HARDWARE/Figures/TurnigyAereoDriveSK3-3536_1400KV.jpg}
    \caption{Turnigy Aereo Drive SK3 3536 1400 KV}
    \label{fig:Hardware drone}
\end{figure}


